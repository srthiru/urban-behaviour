\section{Conclusion}
In this paper, we proposed a system that can extract relevant information from urban datasetes using topological framework, topic modeling and sentiment analysis. Additionally, we presented a visual analytics system to explore the taxi trips and twitter data. The visual interface will help the users to interactively explore the data and extract insights. Furthermore, we hypothesized exists a strong relation between taxi events and twitter conversations in that they contain information about urban behaviors. To support our hypothesis, we presented three case studies. More experiments can be done to further establish this relation. 

The zero-shot classifier provides better results compared to the traditional Topic modeling appraoches, however suffers from performance issues in terms of the time taken for inference. There is a workaround to distill the language network for inference on a set of labels using the teacher model that would make the process faster. We will explore this further.

In future, we want to experiment and analyze different scenarios and case studies to extract more concrete results. Additionally, we plan to make progress with the visual interface so that it can guide users with automatically detected events. Furthermore, better interactions will be provided to the users for ingesting the topic  and sentiment information of the tweets from selected constraints. 

Currently, our visual interface only handles three months of taxi and twitter data. We plan to make the visual system scalable so that it can handle larger amounts of data. Also, our drop off analysis view is visually cluttered and it can create a burden to the cognitive load of the users. Therefore, we will analyze and explore alternative approaches to reduce visual cluttering. Finally, this framework can be extended to pair any urban dataset along with Twitter data, such as the urban noise complaints data to extract events and understand urban behavior.