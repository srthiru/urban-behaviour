\section{Related Work}
% taxi data research
Due to the availability of important information in open taxi data, a lot of research has focused on analyzing them to gain valuable insights. Ge et al. \cite{ge2010energy} proposed a mobile recommender system that recommends taxi drivers potential pick-up and parking positions to maximize business success and minimize energy consumption. Ferreira et al. \cite{ferreira2013visual} proposed a visual system that allows users to make Spatio-temporal queries of taxi trips to visually explore and compare the results. Their proposed system takes into account a large number of taxi data and offers a scalable and efficient solution. Numerous research has been conducted analyzing taxi data from various cities including Lisbon \cite{veloso2011exploratory}, Beijing \cite{liang2012scaling}, Shanghai \cite{peng2012collective}, and so on. These analyses are focused on explaining the taxi pickup and dropoff locations with different city contexts. However, none of these works considered explaining why there might be a sudden change in taxi trips in terms of environmental aspects. In contrast, we concentrate on finding the reason behind this phenomenon with the help of different scenarios such as changes in urban planning, events, and news.  


% %pluto data research
% 	Numerous researchers have tried to analyze different aspects of urban planning of New York City using the rich PLUTO data or with the complementary spatio-temporal Map Pluto data in different use cases. Yuquin et al  \cite{yuqin2021spatio} used PLUTO data along with geotagged social media data to analyze how COVID-19 affected human mobility patterns by land-use types such as residential, commercial, and workplaces, etc. Yang et al.  \cite{yang2021urban} used the correlation between building density in a tax lot area and demographic information and public transit data to understand how urban density, commuting, and economic inequalities affected the infection rates during the pandemic. With our visual system, we aim to look beyond the impact of the pandemic and generalize the effect of social and other events on mobility by also leveraging additional datasets such as taxi data and social media data.
	

% sentiment analysis research
Pang et al. \cite{bo2008opinion} surveyed the significance of sentiment analysis in its early years in and since its research has grown tremendously over the past couple of years. The number of papers published in the field has grown exponentially in recent years, from 37 papers in the year 2000 to 5,699 papers in a matter of 15 years. While earlier algorithms have focused on lexicon-based analysis \cite{maite2011lexicon}, more recent algorithms have focused on machine learning and deep learning. These learning methods have produced excellent results given the pervasiveness of copious amounts of data in modern times. Mäntylä et al \cite{M_ntyl__2018} discussed various modern tools and techniques in sentiment analysis. In addition, there are several APIs providing sentiment analysis capabilities such as Microsoft Text Analytics API, AYLIEN Text Analysis API, Senti API, et cetera. 


%analyzing daily activities
Over the years, there have been many works related to analyzing and understanding daily human activities. Much of this research focuses on data gathered from social networking sites such as Twitter. Social networking sites prove to be a  useful resource for gathering information about human activities as people express and share their thoughts and opinions through them \cite{giachanou2016like}. However, most of the works focus on analyzing public sentiments using Twitter data. In contrast, we direct our analysis on using the sentiment to define and explain certain activities related to the taxi trips.


Overall, none of the existing works of literature have correlated their respective fields with taxi data to gain insights. Our work aims to find significant correlations between each of these above-mentioned domains to determine societal behavior.
