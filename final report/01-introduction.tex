\section{Introduction}
Now-a-days cities offer massive flux of economic and human development opportunities owing to the rapid advancement in urbanization. These opportunities lead to more people migrating to in cities. As more people are bound towards cities, there has been a massive increase in traffic, social events and environmental side-effects, thereby placing more importance on city planning to better manage these factors. Due to the advancement in technology, city officials can automatically collect a large amount of data related to urban behaviors. In recent years, city officials have made these data sets available publicly to research and analyze in order to understand different aspects of urban planning, development and behaviors.

From the different urban datasets available, taxi trips data is one of the most popular and researched as they not only provide valuable information of people's movements through the city in terms of pickups and drop-offs. Hence, it is possible to answer questions about popular places, tourist hotspots and so on by carefully examining taxi data. In particular, in this work, we are interested in identifying such hotspots over time with the help of taxi trip data. After identifying the hotspots we aim to explain the reason for those hotspots. In order to explain such events, we look into social media conversations; specifically Twitter data. Twitter data provides us with a unique perspective as people use the social media platform to discuss their daily activities and share their opinions.

We detail a well rounded framework to detect such hotspots from Taxi data and use this information as spatio-temporal constraints in analyzing the Twitter conversations. In addition, we propose a visual analytics system that can be used to explore these datasets and understand the change in urban behavior.
 

